\documentclass{article}

\usepackage{fancyhdr}


\title{BPP Exercise 1 - print("Hello World")}
\author{A. Hain, M. Nipshagen}
\date{09.04.2018, 8:00}

\makeatletter
\let\thetitle\@title
\let\theauthor\@author
\let\thedate\@date
\makeatother

\pagestyle{fancy}
\fancyhf{}
\fancyhead[L]{\thetitle}
\fancyhead[C]{}
\fancyhead[R]{\theauthor}
\renewcommand{\headrulewidth}{0.4pt} %obere Trennlinie
\fancyfoot[L]{Due: \thedate}
\fancyfoot[R]{\thepage} %Seitennummer
\renewcommand{\footrulewidth}{0.4pt}
\begin{document}

The deadline for this exercise sheet is \textbf{Monday, \thedate.}

\section*{Introductory Words}
In case we have some information that doesn't directly concern the current exercises.

\section{Task}
Lorem ipsum blah

\section{Another Task}
Lorem impsum blah part 2

\end{document}

\usepackage{setspace}

\title{BPP Exercise 8 - 4P}
\author{A. Hain, M. Nipshagen}
\date{04.06.2018, 8:00}

\makeatletter
\let\thetitle\@title
\let\theauthor\@author
\let\thedate\@date
\makeatother


\newcommand\itemsub[1]{
	\begin{itemize}
		\item #1
	\end{itemize}
}

% do not include solutions
% \renewcommand\sol[1]{}


\begin{document}

The deadline for this exercise sheet is \textbf{Monday, \thedate.}

%\section*{Introductory Words}


\section{Warm-Up: The Land Before Time}
Many many years ago, our planet was the habitat of many different types of dinosaurs.\\
We modeled several kinds of them and their behavior (in a manner that may or may
not be a little simplified) in a file named \texttt{dinos.py}.\\
Look at the dino classes given in the file. State for each class the following:
\begin{itemize}
	\item{All superclasses}
	\item{All subclasses}
	\item{All classes it extends}
	\item{All functions it can use}
	\item{All functions it overrides}
\end{itemize}

\section{I want to ride my bicycle, I want to ride my bike}
Implement three classes: \textit{Bike}, \textit{Bicycle} and \textit{Motorbike}.\\
\textit{Bike} is the superclass of both \textit{Bicycle} and \textit{Motorbike}.\\
A \textit{Bike} has a number of seats and a number of gears. It can start being ridden and
end being ridden. Also, one is able to change the gears (to a gear number that
actually exists.)\\
A \textit{Bicycle}, additionally to the \textit{Bike}, has a bell that can be rung.\\
A \textit{Motorbike}, additionally to the \textit{Bike}, has a tank that can be filled.\\
Write a module of the classes according to the description and add another module
that tests them with some values. Don't forget to include proper documentation!




\end{document}
