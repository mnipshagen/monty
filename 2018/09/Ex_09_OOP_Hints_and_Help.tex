\documentclass{article}

\usepackage{fancyhdr}


\title{BPP Exercise 1 - print("Hello World")}
\author{A. Hain, M. Nipshagen}
\date{09.04.2018, 8:00}

\makeatletter
\let\thetitle\@title
\let\theauthor\@author
\let\thedate\@date
\makeatother

\pagestyle{fancy}
\fancyhf{}
\fancyhead[L]{\thetitle}
\fancyhead[C]{}
\fancyhead[R]{\theauthor}
\renewcommand{\headrulewidth}{0.4pt} %obere Trennlinie
\fancyfoot[L]{Due: \thedate}
\fancyfoot[R]{\thepage} %Seitennummer
\renewcommand{\footrulewidth}{0.4pt}
\begin{document}

The deadline for this exercise sheet is \textbf{Monday, \thedate.}

\section*{Introductory Words}
In case we have some information that doesn't directly concern the current exercises.

\section{Task}
Lorem ipsum blah

\section{Another Task}
Lorem impsum blah part 2

\end{document}


\title{BPP Exercise 8 - OOP}
\author{A. Hain, M. Nipshagen}
\date{04.06.2018, 14:00}

\makeatletter
\let\thetitle\@title
\let\theauthor\@author
\let\thedate\@date
\makeatother

\newcommand\itemsub[1]{
	\begin{itemize}
		\item #1
	\end{itemize}
}

\setcounter{secnumdepth}{0}

\begin{document}
The deadline for this exercise sheet is \textbf{Tuesday, \thedate.}
\tableofcontents
\vspace{12pt}\noindent
\textbf{DISCLAIMER:} These are all just suggestions and not necessarily a complete
or the best approach to a solution. It just offers hints, general approaches
and ideas.\\
These are also a lot of pages of one-liners.
\pagebreak

\section{Task 2: Changing Gears to a "valid number"}
Whenever the new gear is not within the number of gears the bike has, throw an exception.

\pagebreak

\section{Task 2: Ringing the Bell}
Ringing the bell can be done by including a new variable or even class as the bell,
but it is also fine to just print something whenever the bell is rung.

\pagebreak

\section{Task 2: What Class Properties and Functions Should I Include?}
Properties correspond to what kinds of things a class \textit{has}.
However, it often happens that there are a few more properties only used for
internal purposes that will need to be added to allow for a certain function to work.\\
Functions correspond to what someone \textit{can do} with the class.

\pagebreak

\section{Task 3: The constants}
The constants should be constants of the module, not the class. Just define them
as variables atop your \texttt{knights} module before defining the class.\\
Technically, they could also be made constants of the class
\pagebreak

\section{Task 3: KnightError}
pass\\\\\\\\\\\\\\\\\\\\\\\\\\\\\\\\\\\\\\\\\\\\\\\\\\\\\\\\\\\\\\\\\\\\\
;)

\pagebreak

\section{Task 3: Knight Class Function Heads \& Returns}
\begin{itemize}
	\item \pythoninline{__init__(self)}: does not return
	\item \pythoninline{alive(self)}: returns boolean
	\item \pythoninline{take_damage(self, damage)}: does not return / could return current health
	\item \pythoninline{light_attack(self)}: returns tuple (damage, crit\_boolean)
	\item \pythoninline{heavy_hit(self)}:  returns tuple (damage, crit\_boolean)
	\item \pythoninline{all_actions(self)}: returns list of function objects
	\item \pythoninline{__str__(self)}: returns string
\end{itemize}
\pagebreak

\section{Task 3: The Health Property}
Remember the \pythoninline{@property} decorator and how to define a private attribute.
You can use the decorator and decorate a function named \texttt{health} with it,
turning it into a property. This function then returns the value of the private health
variable. This getter function is all you need. You don't need a setter function, since
\pythoninline{take_damage} can access the private health property.
\pagebreak

\section{Task 3: The Critical Hit}
You can use  \texttt{random.randint} to calculate whether it was a critical hit.
Generate a random integer between 0 and 99 and check whether it is below the
critical hit chance. If it is, that is a critical hit! So a critical chance of
100 will always result in a hit, while a chance of 1 will only happen if 
\texttt{randint} returns 0.

\pagebreak

\section{Task 3: all\_actions() \& available\_actions}
The trick here is to store the function references in a list. Since you have the
actions \texttt{light\_attack} and \texttt{heavy\_hit}, the list can be defined as\\
\texttt{self.available\_actions = [self.light\_attack, self.heavy\_attack]}\\
Note that the parantheses are missing. This passes the actual \textit{function object}
(They are everywhere...). This function object is this very specific function inside
this very specific instance, so we can do something like\\
\pythoninline{my_knight.all_actions()[0]()}\\
See how we just put a parentheses at the item we got from the list? We are just going
to call the function object that is referenced in the list and it will execute that
specific function in the specific instance of \texttt{my\_knight} we got it from.

\pagebreak

\section{Task 3.1 Bonus: A Subclass}
Remember that you can call \texttt{super().\_\_init\_\_} and afterwards add any
additional attributes you need for your class.

\pagebreak

\section{Task 3.1 Bonus: Description()}
\pythoninline{description()}: returns string \small{\textit{Note there is no self ;)}}

\pagebreak

\section{Task 3.1 Bonus: New Attacks}
If you want to give your knight new attacks, don't forget to update the 
\texttt{available\_actions} list with the newly defined function, so that the
battle code can find and call it.\\

\end{document}
