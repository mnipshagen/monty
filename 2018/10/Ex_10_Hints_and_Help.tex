\documentclass{article}

\usepackage{fancyhdr}


\title{BPP Exercise 1 - print("Hello World")}
\author{A. Hain, M. Nipshagen}
\date{09.04.2018, 8:00}

\makeatletter
\let\thetitle\@title
\let\theauthor\@author
\let\thedate\@date
\makeatother

\pagestyle{fancy}
\fancyhf{}
\fancyhead[L]{\thetitle}
\fancyhead[C]{}
\fancyhead[R]{\theauthor}
\renewcommand{\headrulewidth}{0.4pt} %obere Trennlinie
\fancyfoot[L]{Due: \thedate}
\fancyfoot[R]{\thepage} %Seitennummer
\renewcommand{\footrulewidth}{0.4pt}
\begin{document}

The deadline for this exercise sheet is \textbf{Monday, \thedate.}

\section*{Introductory Words}
In case we have some information that doesn't directly concern the current exercises.

\section{Task}
Lorem ipsum blah

\section{Another Task}
Lorem impsum blah part 2

\end{document}


\title{BPP Exercise 10 - Time, Space and Documentation}
\author{A. Hain, M. Nipshagen}
\date{11.06.2018, 12:00}

\makeatletter
\let\thetitle\@title
\let\theauthor\@author
\let\thedate\@date
\makeatother

\newcommand\itemsub[1]{
	\begin{itemize}
		\item #1
	\end{itemize}
}

\setcounter{secnumdepth}{0}

\begin{document}
The deadline for this exercise sheet is \textbf{Tuesday, \thedate.}
\tableofcontents
\vspace{12pt}\noindent
\textbf{DISCLAIMER:} These are all just suggestions and not necessarily a complete
or the best approach to a solution. It just offers hints, general approaches
and ideas.\\
These are also a lot of pages of one-liners.
\pagebreak

\section{Task 1: How to make a loop that only prints once a second}
You could consider making a loop that runs as long as the second counter is larger than 0.\\
Define some "starting point time" and check if already $\geq$ 1 second has passed over and
over again and only decrease your counter if it did.

\pagebreak

\section{Task 3: BirthDayCalc attributes}
You don't need a lot of attributes. It makes sense to store the birth date and
the date of today.\\
Since you'll need it a lot, you could also consider saving the difference between
those two dates.

\pagebreak

\section{Task 3: How to get the different units}
Get the timedelta between the birth date and today in days and build on top of this
(you need to make a function that returns this anyway ;) ). We'll trust you with
the calculations. Since the leapyears are approximated anyway, it does not actually
have to be accurate to the second of your birth.

\pagebreak

\section{Task 3: Years and months}
We don't recommend to calculate the months based on how many years \textit{as
an integer} have passed because that would only give a very, very approximated
number of months. It might make sense to build an internal private method that
returns the years since birth as a float and then round the result of this function
for \texttt{years\_since\_birth} or use if for further calculation in \texttt{months\_since\_birth}


\end{document}
