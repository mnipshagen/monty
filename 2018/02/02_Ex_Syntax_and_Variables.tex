\documentclass{article}

\usepackage{fancyhdr}


\title{BPP Exercise 1 - print("Hello World")}
\author{A. Hain, M. Nipshagen}
\date{09.04.2018, 8:00}

\makeatletter
\let\thetitle\@title
\let\theauthor\@author
\let\thedate\@date
\makeatother

\pagestyle{fancy}
\fancyhf{}
\fancyhead[L]{\thetitle}
\fancyhead[C]{}
\fancyhead[R]{\theauthor}
\renewcommand{\headrulewidth}{0.4pt} %obere Trennlinie
\fancyfoot[L]{Due: \thedate}
\fancyfoot[R]{\thepage} %Seitennummer
\renewcommand{\footrulewidth}{0.4pt}
\begin{document}

The deadline for this exercise sheet is \textbf{Monday, \thedate.}

\section*{Introductory Words}
In case we have some information that doesn't directly concern the current exercises.

\section{Task}
Lorem ipsum blah

\section{Another Task}
Lorem impsum blah part 2

\end{document}


\title{BPP Exercise 2 - The Very Basics}
\author{A. Hain, M. Nipshagen}
\date{16.04.2018, 8:00}

\makeatletter
\let\thetitle\@title
\let\theauthor\@author
\let\thedate\@date
\makeatother

% uncomment to not include solutions
% \renewcommand\sol[1]{}

\begin{document}

The deadline for this exercise sheet is \textbf{Monday, \thedate.}
%
%\section*{Introductory Words}
%In case we have some information that doesn't directly concern the current exercises.
%
\section{Variable Types}
Assume you're enrolled in a Psychology class.
Your final grade is dependent on your performance in homework, a midterm exam and an experiment as a final project.\\
\emph{Note:} Save your answers for this task in a simple text file.

\subsection{}
You just wrote the midterm and after correcting, your professor thinks it might be handy
to have the exam results for each student digitally accessible. Which variable types would you use
for variables to save the following things in? Explain your answers.
\begin{enumerate}
\item the student's name\sol{String, since we need to represent a bunch of characters}
\item the student's matriculation number\sol{Int or String, both work. If the system is like the UOS one, int is just fine.}
\item the student's exam grade\sol{Float, unviersity grades are (often) floats}
\item if the student passed the exam or not\sol{Boolean, there are only two options.}
\end{enumerate}

\subsection{}
Fast foward some time, you need to conduct the experiment. In the experiment you
present subjects with a short stimulus and measure their score based on reaction.
Which variable types would you use for the following variables in your experiment?
\begin{enumerate}
	\item subject id\sol{Int, subjects are usually anonymised and represented by a number}
	\item subject age\sol{Int, for children we can go by month, for adults months are usually not relevant}
	\item subject sex\sol{Boolean, sex is not gender and since we need to have a representative sample outliers are in most cases excluded}
	\item stimulus onset in ms\sol{Float|Int, depending on the accuracy needed}
	\item reaction time in ms\sol{Float|Int, depending on the accuracy needed}
	\item subject test group (A, B, or C)\sol{String or Int, You can correlate a,b,c with 0,1,2, both work}
	\item subject scores\sol{Int or Float, depending on calculation of score}
	\item subject overall score\sol{Float, often an average or other statistical meausre, which are mostly non integers}
\end{enumerate}

\section{Some Math}
In this task, you will have to write two small Python programs. Please save each
of them in a separate file.

\subsection{Finishing up on that experiment}
Finishing up your experiment you need to calculate an overall performance.
Each subject in the end has 4 scores, and you calculate their overall performance
with the following formula $$ \dfrac{4*Score1 + Score2 + 0.5*Score3 + \sqrt{Score4}}{2\pi} $$
\emph{Note:} Totally arbitrary forumla with probably no real world use.\\
You can approximate \(\pi \) with 3.14 or you can use the following to get pi in python:
\begin{python}
import math.pi


print(pi)
\end{python}

Write a function that calculates and returns the overall performance of a subject when you pass it the five scores.
\cprotect\sol{
\begin{python}
import math


def overall_score(score1, score2, score3, score4, score5):
	# **05 is the same as math.sqrt
	# saving the result in a variable to avoid overly lenghty code
	tmp = 4*score1 + score2 + 0.5*score3 + score4**0.5
	return tmp/(2*math.pi)
\end{python}
}

\subsection{The Prof's turn}
After you completed your Psychology class, your professor now wants to determine your final grade.
It is composed of your homework grade by 20\%, your midterm grade by 30\% and your final
project grade by 50\%.\\
Write a Python script which will calculate the final grade out of those three grades
and print the result. Try to use suitable names for your variables and test your program with
a few different values.

\cprotect\sol{
\begin{python}
def final_grade(homework, midterm, project):
	return 0.2*homework + 0.3*midterm + 0.5*project
\end{python}
}

\subsection{Time for Cake}
After passing the class, you want to treat yourself by baking a chocolate cake.
You find a nice recipe for the dough online.
The quantities of the ingredients are perfect to make one cake in a circular cake mould with a diameter
of 28cm and a height of 8cm. However, you only have a cake mould with a diameter of 24cm and a height of 6.5cm.\\
Write a script that contains a function to determine by which factor you need to multiply the cake ingredients
to make the perfect amount of dough for your smaller cake moulds and prints the result.\\
What is the factor if you only have a cake mould with a 20cm diameter/6cm height or 18cm diameter/6cm height?

\cprotect\sol{
\begin{python}
# technically pi is not needed for the calculation as it is reduced in the division
import math

# take the height and diameter of the recipe and the ones that you need to fit
# then returns the factor by which the ingredients need to be multiplied
def recipe_factor(height, diameter, orig_height=28, orig_diameter=8):
	orig_volume = orig_height * (orig_diameter / 2)**2 * math.pi
	new_volume = height * (diameter / 2)**2 * math.pi
	return new_volume / orig_volume
\end{python}
}

\begin{center}
\includegraphics[scale=0.6]{Ex_2_Cake}
\end{center}

\end{document}
