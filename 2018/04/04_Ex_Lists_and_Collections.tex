\documentclass{article}

\usepackage{fancyhdr}


\title{BPP Exercise 1 - print("Hello World")}
\author{A. Hain, M. Nipshagen}
\date{09.04.2018, 8:00}

\makeatletter
\let\thetitle\@title
\let\theauthor\@author
\let\thedate\@date
\makeatother

\pagestyle{fancy}
\fancyhf{}
\fancyhead[L]{\thetitle}
\fancyhead[C]{}
\fancyhead[R]{\theauthor}
\renewcommand{\headrulewidth}{0.4pt} %obere Trennlinie
\fancyfoot[L]{Due: \thedate}
\fancyfoot[R]{\thepage} %Seitennummer
\renewcommand{\footrulewidth}{0.4pt}
\begin{document}

The deadline for this exercise sheet is \textbf{Monday, \thedate.}

\section*{Introductory Words}
In case we have some information that doesn't directly concern the current exercises.

\section{Task}
Lorem ipsum blah

\section{Another Task}
Lorem impsum blah part 2

\end{document}



\title{BPP Exercise 4 - Lists and Collections}
\author{A. Hain, M. Nipshagen}
\date{30.04.2018, 8:00}

\makeatletter
\let\thetitle\@title
\let\theauthor\@author
\let\thedate\@date
\makeatother

% Defining the Left Pointy Bracket and the Right Pointy Bracket, so they look nicer
\newcommand\lpb{\small{<}}
\newcommand\rpb{\small{>}}
% sublists are annoying
\newcommand\SubPoint[1]{
  \begin{itemize}
    \item #1
  \end{itemize}
  }
  
% do not include solutions
% \renewcommand\sol[1]{} \renewcommand\SubPoint[1]{}


\begin{document}

The deadline for this exercise sheet is \textbf{Monday, \thedate.}
%
%\section*{Introductory Words}
%In case we have some information that doesn't directly concern the current exercises.
%
\section{Vector Math}
We can model vectors with tuples and lists. Python however is not equipped with the tools to do vector math, but we can write the functions for this ourselves!\\
Write a script \texttt{vectors.py} which defines the following functions:
\begin{itemize}
  \item \texttt{add(x, y)}: Adds $x$ and $y$ such that the result follows $z_i = x_i + y_i$.
    \cprotect\sol{
\begin{python}
def add(x, y):
      """Adds up the vectors x and y"""
      return [x[i] + y[i] for i in range(len(x))]
\end{python}
    }
  \item \texttt{sub(x, y)}: Subtracts $y$ from $x$ such that the result follows $z_i = x_i - y_i$.
    \cprotect\sol{
\begin{python}
def sub(x, y):
    """returns the result of subtracting y from x"""
		return [x[i] - y[i] for i in range(len(x))]
\end{python}
    }
  \item \texttt{dot(x, y)}: Calculates the scalar (dot) product (or inner product) of $x$ and $y$. The dot product $\lpb x,y\rpb$ is defined as $\lpb x,y\rpb = \sum\limits_{i=1}^{N}x_iy_i$.
\cprotect\sol{\begin{python}
def dot(x, y):
    """Returns the inner product of the two vectors"""
		sum([x[i]y[i] for i in range(len(x))])
	\end{python}}
  \item \texttt{angle(x, y)}: Calculates the angle $\theta$ between the between $x$ and $y$. The angle can be found by using an alternative definition of the dot product: $\lpb x,y\rpb = \|x\|\|y\| \cos\theta$, which we can then solve for $\theta$ and we get $\theta = \arccos\dfrac{\lpb x,y\rpb}{\|x|\|y\|}$.\\
  \emph{Note:} $\|x\|$ where $x$ is a vector is called the vector norm and is calculated by $\sqrt{<x,x>}$. You can find the arccos function in the \texttt{math} package, which we previously used for pi. The function is called \texttt{acos}.
\cprotect\sol{\begin{python}
import math


def angle(x, y):
    """Returns the angle between the two vectors in radien"""
    divisor = math.sqrt(dot(x, x)) * math.sqrt(dot(y, y))
    return math.acos(dot(x, y) / divisor)
	\end{python}}
  \item \texttt{pdist(x, y, **kwargs)}: The distance between the two points $x$ and $y$. Your \textit{kwargs} should recognise the keywords \texttt{metric} and \texttt{p}. Your \texttt{metric} keyword should be able to take one of the values \texttt{'euclidean', 'minkowski', 'cityblock'}, and \texttt{p} can take any integer greater or equal than one. The distance is calculated depending on the metric and should default to \texttt{euclidean}. The distances are calculated as follows:
    \begin{itemize}
      \item \texttt{euclidean}: $d_{euclid} (x,y) = \sqrt[\uproot{2}2]{\sum\limits_{i=1}^N \|x_i-y_i\|^2}$.
      \item \texttt{minkowski}: $d_{minkowsky} (x,y) = \sqrt[\uproot{2}\texttt{p}]{\sum\limits_{i=1}^N \|x_i-y_i\|^\texttt{p}}$.
      \item \texttt{cityblock}: $d_{cityblock} (x,y) = \sum\limits_{i=1}^N \|x_i-y_i\|$\\
      (\emph{Note:} $\|x\|$ is the absolute value.)
    \end{itemize}
    See figure \ref{fig:metrics} for a nicer visualisation.\\
    Don't be afraid of this. It looks scary at first, but it is not beyond the scope of what you already learned.
\cprotect\sol{\begin{python}
import math


def pdist(x, y, **kwargs):
    """
    Calculates the distance between x and  y using the given metric.

    The default metric used for distance calculation is the euclidean metric. Other options are 'cityblock' and 'minkowski'. If using minkowski distance, the parameter p >= 1 can be supplied.
    """
    # check whether metric is given, and use euclidean as default
    metric = kwargs['metric'] if 'metric' in kwargs else 'euclidean'
     if metric == 'minkowski':
        p = kwargs['p'] if 'p' in kwargs else 2
    if metric == 'euclidean':
        squared = [abs(z)**2 for z in sub(x,y)]
        return math.sqrt(sum(squared))
    elif metric == 'cityblock':
        diff = [abs(z) for z in sub(x, y)]
        return sum(diff)
    elif metric == 'minkowski':
        if p < 1:
             return None
        to_p = [abs(z)**p for z in sub(x, y)]
        return sum(to_p) ** (1/p)
    else:
        return None

\end{python}
}
  \item \textbf{BONUS:} \texttt{outer(x, y)}: calculates the outer product of two vectors. The outer product of two vectors is the tensor product of the two. For two vectors with size $n$ the outer product will yield a matrix of size $(n, n)$.
  The outer product is calculated as $C_{i_j} = x_iy_j$.
  \cprotect\sol{\begin{python}
def outer(x, y):
    """Returns a list of lists representinx a matrix calculated from the two vectors, each list represents one row"""
    # one liner:
    # return [[xi * yi for yi in y] for xi in x]
    result = []
    for i in range(len(x)):
        row = []
        for j in range(len(y)):
          row.append(x[i] * y[j])
        result.append(row)
		return result
	\end{python}}
\end{itemize}
You can assume that all vectors have the same size. To verify your function works correctly you can test it with the following values:\\
$a = (1, 2, 3), b = (4, 5, 6), c = [0, 1, 0, 0, 1], d = [1.5, 2.5, 3.5, 4.5, 5.5]$\\
\begin{tabular}{|ll|}
  \hline
  Call & Result \\
  \hline
  \texttt{add(a,b)} & (5, 7, 9)\\
  \texttt{add(b,a)} & (5, 7, 9)\\
  \texttt{sub(a,b)} & (-3, -3, -3)\\
  \texttt{dot(c,d)} & 8.0\\
  \texttt{angle(a,b)} & approx. 0.23\\
  \texttt{pdist(a,b)} & approx. 5.20\\
  \texttt{pdst(a,b, metric='cityblock')} & 9\\
  \texttt{pdist(c,d, metric='minkowski', p=3)} & 6.14\\
  \texttt{outer(a,b)} & $\left[\begin{matrix} 4 & 5 & 6\\ 8 & 10 & 12\\ 12 & 15 & 18 \end{matrix}\right]$\\
  \hline
\end{tabular}
\begin{figure}
  \includegraphics[scale=0.5]{metriken}
  \label{fig:metrics}
  \caption{Blue, red, and yellow are examples for the cityblock distance (it looks like following the streets of a grid city) and green is the euclidean distance.}
  \small{Taken from \url{https://en.wikipedia.org/wiki/Taxicab_geometry#/media/File:Manhattan_distance.svg}}
  
\end{figure}

\section{Which to What}
We will give you a few examples of datasets. Which collection type would you use to save this data in?\\
Please explain your answers.
\begin{itemize}
\item options a user can click on in a program menu
\SubPoint{\sol{List}}
  \item a country's name, population and capital
    \SubPoint{\sol{Tuple or Dict}}
  \item food ingredients
    \SubPoint{\sol{List}}
  \item data about a music album
    \SubPoint{\sol{Dict}}
  \item people who visit Basic Programming in Python or Scientific Programming in Python
    \SubPoint{\sol{Set}}
  \item IDs of upcoming orders in an online shop
    \SubPoint{\sol{List}}
  \item nicknames and account information in a forum (e-mail address, password, real name [optional], birth date [optional], ...)
    \SubPoint{\sol{Dict}}
\end{itemize}

\section{The Transform}
In the file \texttt{my\_collection.py} you will find two lists defined: \texttt{subjects} and \texttt{attributes}. The first list contains subject ids, wheras the second one contains attributes corresponding to the subject ids. So the subject at index 0 of \texttt{subjects} has the attribute at index 0 of \texttt{attributes} and the subject at index 9 has the attribute at index 9. You get the idea.\\
You may notice that each subject id appears multiple times. Your task now is to create a function to create one dictionary which uses the subject id as a key and a \textit{list} of attributes as the value.
\cprotect\sol{
\begin{python}
# the given lists
all_subjects = [0, 0, 1, 1, 1, 1, 2, 2, 2, 3, 3, 3, 3, 4, 4, 4, 5, 5, 5, 6, 6, 6, 6, 6, 6, 7, 7, 7, 7, 7, 7, 8, 8, 8, 8, 8, 9, 9, 9]
all_attributes = ['Materialistic', 'Neat', 'Active', 'Welcoming', 'Creative', 'Ambitious', 'Geek', 'Welcoming', 'Neat', 'Creative', 'Geek', 'Quiet', 'Shy', 'Neat', 'Ambitious', 'Adventurous', 'Active', 'Welcoming', 'Adventurous', 'Neat', 'Ambitious', 'Excitable', 'Active', 'Welcoming', 'Quiet', 'Excitable', 'Ambitious', 'Adventurous', 'Quiet', 'Geek', 'Active', 'Spiritual', 'Quiet', 'Excitable', 'Materialistic', 'Geek', 'Welcoming', 'Excitable', 'Adventurous']

# This is the function you need to implement
def clean_up(subjects, attributes):
  """
  This function takes a list of subject ids which correspond to attributes
  in the second list, and forms a dictionary out of them, with the unique
  subject id as the key and a list of their attributes as the corresponding
  value.
  """
  # and as a one-liner:
  # return {sub : attributes[subjects.index(sub):subjects.index(sub)+subjects.count(sub)] for sub in set(subjects)}
  # create the empty dict that we will keep adding the stuff into one by one
  subject_dict = dict()
  # idx is the counter going from 0 to 38
  # using enumerate saves as the line: subj_id = subjects[idx]
  for idx, subj_id in enumerate(subjects):
  # if this is the first time we encounter this subject id, add it to the dict
  # the value is an empty list for now, we will now add all the attributes
    if subj_id not in subject_dict:
      subject_dict[subj_id] = []
    # add the current attribute to the list of the subject
    subject_dict[subj_id].append(attributes[idx])

  return subject_dict

# So nice, tidy and clean.
subject_dict = clean_up(all_subjects, all_attributes)
\end{python}
}
\end{document}
