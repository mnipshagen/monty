\documentclass{article}

\usepackage{fancyhdr}


\title{BPP Exercise 1 - print("Hello World")}
\author{A. Hain, M. Nipshagen}
\date{09.04.2018, 8:00}

\makeatletter
\let\thetitle\@title
\let\theauthor\@author
\let\thedate\@date
\makeatother

\pagestyle{fancy}
\fancyhf{}
\fancyhead[L]{\thetitle}
\fancyhead[C]{}
\fancyhead[R]{\theauthor}
\renewcommand{\headrulewidth}{0.4pt} %obere Trennlinie
\fancyfoot[L]{Due: \thedate}
\fancyfoot[R]{\thepage} %Seitennummer
\renewcommand{\footrulewidth}{0.4pt}
\begin{document}

The deadline for this exercise sheet is \textbf{Monday, \thedate.}

\section*{Introductory Words}
In case we have some information that doesn't directly concern the current exercises.

\section{Task}
Lorem ipsum blah

\section{Another Task}
Lorem impsum blah part 2

\end{document}


\title{BPP Exercise 11 - NumPy and Matplotlib}
\author{A. Hain, M. Nipshagen}
\date{18.06.2018, 10:00}

\makeatletter
\let\thetitle\@title
\let\theauthor\@author
\let\thedate\@date
\makeatother

\newcommand\itemsub[1]{
	\begin{itemize}
		\item #1
	\end{itemize}
}

\setcounter{secnumdepth}{0}

\begin{document}
The deadline for this exercise sheet is \textbf{Tuesday, \thedate.}
\tableofcontents
\vspace{12pt}\noindent
\textbf{DISCLAIMER:} These are all just suggestions and not necessarily a complete
or the best approach to a solution. It just offers hints, general approaches
and ideas.\\
These are also a lot of pages of one-liners.
\pagebreak

\section{Crossed Out - Functions}
For each of the mathematical functions, create one Python function. The broadcasting
of numpy will handle the multiplication with a scalar as well as the power function.
This means you kind of just have to write down the math formula in Python and it 
should work with very little adjustments.

\pagebreak

\section{Markers all around - np.normal}
Next to the center and the standard deviation, you can pass a shape parameter to
\verb|np.normal|, which will give you then back a filled ndarray in that shape.

\pagebreak

\section{Markers all around - histogram, bins and width}
The histogram function \verb|np.hist| takes the parameters bins, which can be 
an integer, which then represents how many bins to use, and rwidth, which scales
the width of each bar of the histogram with 1 being the full width. 

\pagebreak

\section{Who is that plot?!}
You can use the output from \verb|get_x_y| directly as input to \verb|np.scatter|,
and well you should. The function accepts the s parameter, which scales the markers
of the plot.

\pagebreak

\section{Wavy Waves - Reading the plot}
From the plot you can read the height and the width of the sine waves. They are called
the amplitude and the period, respectively. Using this you can infer the data, that
you need to plot those waves.

\pagebreak

\section{Wavy Waves - The sine function}
When you got the amplitude and the period of the sine waves you can use them in the
sine formula like $Amplitude * sin(period * x)$. The \verb|np.sin| function can
apply sine to a whole array.


\end{document}
