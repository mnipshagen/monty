\documentclass{article}

\usepackage{fancyhdr}


\title{BPP Exercise 1 - print("Hello World")}
\author{A. Hain, M. Nipshagen}
\date{09.04.2018, 8:00}

\makeatletter
\let\thetitle\@title
\let\theauthor\@author
\let\thedate\@date
\makeatother

\pagestyle{fancy}
\fancyhf{}
\fancyhead[L]{\thetitle}
\fancyhead[C]{}
\fancyhead[R]{\theauthor}
\renewcommand{\headrulewidth}{0.4pt} %obere Trennlinie
\fancyfoot[L]{Due: \thedate}
\fancyfoot[R]{\thepage} %Seitennummer
\renewcommand{\footrulewidth}{0.4pt}
\begin{document}

The deadline for this exercise sheet is \textbf{Monday, \thedate.}

\section*{Introductory Words}
In case we have some information that doesn't directly concern the current exercises.

\section{Task}
Lorem ipsum blah

\section{Another Task}
Lorem impsum blah part 2

\end{document}



\title{BPP Exercise 3 - Directing the Flow}
\author{A. Hain, M. Nipshagen}
\date{23.04.2018, 8:00}

\makeatletter
\let\thetitle\@title
\let\theauthor\@author
\let\thedate\@date
\makeatother

% do not include solutions
% \renewcommand\sol[1]{}

\begin{document}

The deadline for this exercise sheet is \textbf{Monday, \thedate.}
%
%\section*{Introductory Words}
%In case we have some information that doesn't directly concern the current exercises.
%
\section{Boolean Operators}
Determine the truth values of the following boolean operators for each configuration
of truth values as given in the tables

\subsection{}
\begin{tabular}{| c | c | c | c |}
  \hline
  \textbf{a} & \textbf{b} & \textbf{c} & \textbf{(b or c) and (a or c)} \\
  \hline
  true & true & true & \sol{true} \\
  \hline
  true & true & false & \sol{true} \\
  \hline
  true & false & true & \sol{true} \\
  \hline
  true & false & false & \sol{false} \\
  \hline
  false & true & true & \sol{true} \\
  \hline
  false & true & false & \sol{false} \\
  \hline
  false & false & true & \sol{true} \\
  \hline
  false & false & false & \sol{false} \\
  \hline
\end{tabular}

\subsection{}
\begin{tabular}{| c | c | c | c |}
  \hline
  \textbf{a} & \textbf{b} & \textbf{c} & \textbf{a or (b and c) and (not c or not a)} \\
  \hline
  true & true & true & \sol{false} \\
  \hline
  true & true & false & \sol{true} \\
  \hline
  true & false & true & \sol{false} \\
  \hline
  true & false & false & \sol{true} \\
  \hline
  false & true & true & \sol{true} \\
  \hline
  false & true & false & \sol{false} \\
  \hline
  false & false & true & \sol{false} \\
  \hline
  false & false & false & \sol{false} \\
  \hline
\end{tabular}


\section{Warm up -- Prof Strikes Again}
Remember the task from last week? Now that your prof computed your numerical grade, the
functionality of this program shall be extended by automatically determining whether
someone passed the class or not based on this grade.\\
Write a function \texttt{passed} that takes a grade
as a parameter and returns whether the student passed or failed.

\cprotect\sol{
\begin{python}
def passed(grade):
  return grade <= 4.0
\end{python}
}

\section{Loops}

\subsection{N Bottles of Beer}
Similar to Hello World programs, 99 bottles programs give us an idea of how
a programming language looks as they show the basic loop concepts.
The 99 bottles program “sings” a little song which goes like this:\\\\
\textit{
99 bottles of beer on the wall, 99 bottles of beer. Take one down and
pass it around, 98 bottles of beer on the wall.\\
98 bottles of beer on the wall, 98 bottles of beer. Take one down and
pass it around, 97 bottles of beer on the wall.\\
. . .\\
1 bottle of beer on the wall, 1 bottle of beer. Take one down and
pass it around, no more bottles of beer on the wall.\\\\
}
Write a function \texttt{n\_bottles(n)} in the script \texttt{n\_bottles.py} which
sings the song starting with n bottles instead of 99. If n is bigger than 99 or
smaller than 5 print a message that you want to sing a funnier song than n bottles
(of course replace n with the current n). Your final result may also structure
the verses in a different layout.

\cprotect\sol{
\begin{python}
def bottles(n):
  return ('1 bottle' if (n == 1) else str(n) + ' bottles') + ' of beer'

def n_bottles(n):
  if not 5 <= n <= 99:
    print('I want to sing funnier songs than "'
      + bottles(n) + '".\n')
    return
  while n > 0:
    print(bottles(n) + ' on the wall,\n ' + bottles(n) + '.')
    n = n - 1
    print('Take one down and pass it around,\n '
      + bottles(n if n > 0 else 'no more') + ' on the wall.\n')

n_bottles(2)
n_bottles(1013)
n_bottles(5)
\end{python}
Output:

% add output. make format pretty.
}


\subsection{}

\end{document}
