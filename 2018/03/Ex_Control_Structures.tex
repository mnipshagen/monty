\documentclass{article}

\usepackage{fancyhdr}


\title{BPP Exercise 1 - print("Hello World")}
\author{A. Hain, M. Nipshagen}
\date{09.04.2018, 8:00}

\makeatletter
\let\thetitle\@title
\let\theauthor\@author
\let\thedate\@date
\makeatother

\pagestyle{fancy}
\fancyhf{}
\fancyhead[L]{\thetitle}
\fancyhead[C]{}
\fancyhead[R]{\theauthor}
\renewcommand{\headrulewidth}{0.4pt} %obere Trennlinie
\fancyfoot[L]{Due: \thedate}
\fancyfoot[R]{\thepage} %Seitennummer
\renewcommand{\footrulewidth}{0.4pt}
\begin{document}

The deadline for this exercise sheet is \textbf{Monday, \thedate.}

\section*{Introductory Words}
In case we have some information that doesn't directly concern the current exercises.

\section{Task}
Lorem ipsum blah

\section{Another Task}
Lorem impsum blah part 2

\end{document}



\title{BPP Exercise 3 - Directing the Flow}
\author{A. Hain, M. Nipshagen}
\date{23.04.2018, 8:00}

\makeatletter
\let\thetitle\@title
\let\theauthor\@author
\let\thedate\@date
\makeatother

% do not include solutions
% \renewcommand\sol[1]{}

\begin{document}

The deadline for this exercise sheet is \textbf{Monday, \thedate.}
%
%\section*{Introductory Words}
%In case we have some information that doesn't directly concern the current exercises.
%
\section{Boolean Operators}
Determine the truth values of the following boolean operators for each configuration
of truth values as given in the tables

\subsection{}
\begin{tabular}{| c | c | c | c |}
  \hline
  \textbf{a} & \textbf{b} & \textbf{c} & \textbf{(b or c) and (a or c)} \\
  \hline
  true & true & true & \sol{true} \\
  \hline
  true & true & false & \sol{true} \\
  \hline
  true & false & true & \sol{true} \\
  \hline
  true & false & false & \sol{false} \\
  \hline
  false & true & true & \sol{true} \\
  \hline
  false & true & false & \sol{false} \\
  \hline
  false & false & true & \sol{true} \\
  \hline
  false & false & false & \sol{false} \\
  \hline
\end{tabular}

\subsection{}
\begin{tabular}{| c | c | c | c |}
  \hline
  \textbf{a} & \textbf{b} & \textbf{c} & \textbf{a or (b and c) and (not c or not a)} \\
  \hline
  true & true & true & \sol{true} \\
  \hline
  true & true & false & \sol{true} \\
  \hline
  true & false & true & \sol{true} \\
  \hline
  true & false & false & \sol{true} \\
  \hline
  false & true & true & \sol{true} \\
  \hline
  false & true & false & \sol{false} \\
  \hline
  false & false & true & \sol{false} \\
  \hline
  false & false & false & \sol{false} \\
  \hline  
\end{tabular}

\subsection{}
\begin{tabular}{| c | c | c | c |}
  \hline
  \textbf{a} & \textbf{b} & \textbf{c} & \textbf{not(not(b and not(c or a)))} \\
  \hline
  true & true & true & \sol{false} \\
  \hline
  true & true & false & \sol{false} \\
  \hline
  true & false & true & \sol{false} \\
  \hline
  true & false & false & \sol{false} \\
  \hline
  false & true & true & \sol{false} \\
  \hline
  false & true & false & \sol{true} \\
  \hline
  false & false & true & \sol{false} \\
  \hline
  false & false & false & \sol{false} \\
  \hline
\end{tabular}

\section{Warm up -- Prof Strikes Again}
Remember the task from last week? Now that your prof computed your numerical grade, the
functionality of this program shall be extended by automatically determining whether
someone passed the class or not based on this grade.\\
Write a function \texttt{passed} that takes a grade
as a parameter and returns whether the student passed or failed.

\cprotect\sol{
\begin{python}
def passed(grade):
  """Returns True if the grade is a passing one"""
  return grade <= 4.0
\end{python}
}

\section{Loops}

\subsection{N Bottles of Beer}
Similar to Hello World programs, 99 bottles programs give us an idea of how
a programming language looks as they show the basic loop concepts.
The 99 bottles program “sings” a little song which goes like this:\\\\
\textit{
99 bottles of beer on the wall, 99 bottles of beer. Take one down and
pass it around, 98 bottles of beer on the wall.\\
98 bottles of beer on the wall, 98 bottles of beer. Take one down and
pass it around, 97 bottles of beer on the wall.\\
. . .\\
1 bottle of beer on the wall, 1 bottle of beer. Take one down and
pass it around, no more bottles of beer on the wall.\\\\
}
Write a function \texttt{n\_bottles(n)} in the script \texttt{n\_bottles.py} which
sings the song starting with n bottles instead of 99. If n is bigger than 99 or
smaller than 5 print a message that you want to sing a funnier song than n bottles
(of course replace n with the current n). Your final result may also structure
the verses in a different layout.

\cprotect\sol{
\begin{python}
def bottles(n):
    """Formats plural according to number of bottles."""
    return ('1 bottle' if n == 1 else str(n) + ' bottles') + ' of beer'

def n_bottles(n):
    """
    Go through the bottles of beer on the wall, and pass them around.

    Calls `bottles(n)` for formatting, and reduces n by 1 each iteration
    """
    if 5 <= n <= 99:
        while(n > 0):
            print(bottles(n) + ' on the wall,\n  ' + bottles(n) + '.')
            n = n - 1
            print('Take one down and pass it around,\n  ' +
                # conditional expression to determine whethere there are bottles left
                bottles(n if n > 0 else 'no more') + 
                ' on the wall.\n')
    else:
        print('I want to sing funnier songs than "' + bottles(n) + '".\n')


n_bottles(3)
n_bottles(1011)
n_bottles(5)
\end{python}
Output:
\texttt{
\small{\\
I want to sing funnier songs than "3 bottles of beer".\\\\
I want to sing funnier songs than "1011 bottles of beer".\\\\
5 bottles of beer on the wall,\\
\indent 5 bottles of beer.\\
Take one down and pass it around,\\
\indent 4 bottles of beer on the wall.\\\\
4 bottles of beer on the wall,\\
\indent 4 bottles of beer.\\
Take one down and pass it around,\\
\indent 3 bottles of beer on the wall.\\\\
3 bottles of beer on the wall,\\
\indent 3 bottles of beer.\\
Take one down and pass it around,\\
\indent 2 bottles of beer on the wall.\\\\
2 bottles of beer on the wall,\\
\indent 2 bottles of beer.\\
Take one down and pass it around,\\
\indent 1 bottle of beer on the wall.\\\\
1 bottle of beer on the wall,\\
\indent 1 bottle of beer.\\
Take one down and pass it around,\\
\indent no more bottles of beer on the wall.\\
}
}
}


\subsection{Return of the turtle}
\FloatBarrier
You hopefully remember the turtle from the first week -- you drew a Saint Nicholas' house with it.
This time we will draw even more houses! And some trees to keep them company.\\
Write a script \texttt{turtle\_world.py}. Develop a function \texttt{draw\_house} that draws a house --
it doesn't need to be the Saint Nicholas' house, but you can recylce your code if you like.
Then write a function \texttt{draw\_tree{height}} that draws a tree of a given "height".
The height serves a double purpose as the branching factor of the tree. Your tree is going to be a fractal,
which means that it will be a pattern that repeats itself recursively.

\begin{figure}[h]
  \includegraphics[scale=0.8]{recursive_tree}
  \caption{Example Tree}
  \label{fig:tree:1}
\end{figure}


\noindent To build the tree follow this algorithm:\\
\begin{lstlisting}
To draw a tree with height h:
  If height h is 0, stop.
  Draw a line of length L * h.
  Rotate left by angle A.
  Draw a tree of height h - 1.
  Rotate right by angle 2A.
  Draw a tree of height h - 1.
  Rotate left by angle A.
  Move back to the beginning of the line.
\end{lstlisting}

\noindent The resulting tree should look similar to Figure \ref{fig:tree:1}. Choose $A$ and $L$ as you like, be creative!\\
Now draw a simple landscape. Draw a house, a small tree, a big tree, a small tree, and repeat this pattern a couple of times
(Figure \ref{fig:world:flat}). Or build a different one. Just make it repetitive (Yes, use loops).

\begin{figure}[h]
	\includegraphics[width=\textwidth]{flat_world}
	\caption{Flat World}
	\label{fig:world:flat}
\end{figure}

\noindent\textbf{Bonus}: Can you build a round world like in Figure \ref{fig:world:round}?

\begin{figure}[h]
	\includegraphics[width=\textwidth]{round_world}
	\caption{Round World}
	\label{fig:world:round}
\end{figure}

\FloatBarrier
\cprotect\sol{
\begin{python}
# pylint: disable=E1101
import time
import turtle


# constants for all draw functions for a nice and consistent look
LENGTH = 6
ANGLE = 35

def draw_tree(height):
    """
    Draws a fractal tree with `height` repetitions.
    
    `height` defines the overall height of the tree is also responsible for
    the length of each branch in every iteration
    """
    if height == 0:
        return
    # left branch
    turtle.forward(LENGTH * height)
    turtle.left(ANGLE)
    draw_tree(height - 1) # drawing the little tree at the end of the branch
    # right branch
    turtle.right(2 * ANGLE)
    draw_tree(height - 1)  # drawing the little tree at the end of the branch
    turtle.left(ANGLE)
    turtle.backward(LENGTH * height)


def draw_house():
    """This draws a nice and simple house!"""
    # the dimensions of our house
    height = 5 * LENGTH
    width = 7 * LENGTH
    roofside = (width ** 2 / 2) ** (1 / 2)

    # left wall
    turtle.forward(height)
    # roof
    turtle.right(45)
    turtle.forward(roofside)
    turtle.right(90)
    turtle.forward(roofside)
    turtle.right(45)
    # right wall
    turtle.forward(height)
    turtle.right(90)
    # bottom line
    turtle.forward(width)
    turtle.right(90)


def draw_world(curvature_step=0):
    """
    This draws a turtle world.
    
    The curvature step is relevant for drawing a round world.
    The higher the curvature step is, the smaller our circle will be.
    
    Each village will consist of one house and 3 trees, with one being taller.
    """
    if curvature_step > 0: # this ensures we are going full circle
        villages = 360 // 4 // curvature_step
    else: # 5 villages for our flat world
        villages = 5

    # the _ is called an anonymous variable, since we don't use it anyway
    # we don't need to give it a name. It just acts as a counter.
    for _ in range(villages):
        prepare_drawing()
        draw_house()
        finish_drawing()

        turtle.right(curvature_step)
        turtle.forward(LENGTH * 11)

        # and draw the three trees
        for j in range(3):
            prepare_drawing()
            # the middle one will be 5 high, since we iterate over 0,1,2
            # and only for 1 modulo 2 is 1 returned.
            draw_tree(3 + j % 2 * 2)
            finish_drawing()

            turtle.right(curvature_step)
            turtle.forward(LENGTH * 3)

        turtle.forward(LENGTH)


def init():
    """set up the turtle parameters"""
    turtle.reset()
    turtle.shape('turtle')
    turtle.speed('fastest')
    turtle.up()


def prepare_drawing():
    """move the pen down to actually draw and make turtle upright"""
    turtle.down()
    turtle.left(90)


def finish_drawing():
    """move pen up to stop drawing and return turtle to axis"""
    turtle.right(90)
    turtle.up()


def draw_flat_world():
    """wrapper to start drawing a flat world with 0 curvature"""
    init()
    turtle.goto(-300, 0)

    draw_world()
    turtle.goto(0,0)


def draw_round_world(curvature_step=5):
    """wrapper to draw a curved world with a default curvature step of 5"""
    init()
    turtle.goto(0, 300)
    draw_world(curvature_step)
    turtle.goto(0,0)


def draw():
    """Draw the flat world. Rest shortly to marvel at it. Draw round world."""
    draw_flat_world()
    time.sleep(3)
    draw_round_world()
    turtle.done()

# Start the party!
draw()

\end{python}
}

\end{document}
