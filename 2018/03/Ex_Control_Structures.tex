\documentclass{article}

\usepackage{fancyhdr}


\title{BPP Exercise 1 - print("Hello World")}
\author{A. Hain, M. Nipshagen}
\date{09.04.2018, 8:00}

\makeatletter
\let\thetitle\@title
\let\theauthor\@author
\let\thedate\@date
\makeatother

\pagestyle{fancy}
\fancyhf{}
\fancyhead[L]{\thetitle}
\fancyhead[C]{}
\fancyhead[R]{\theauthor}
\renewcommand{\headrulewidth}{0.4pt} %obere Trennlinie
\fancyfoot[L]{Due: \thedate}
\fancyfoot[R]{\thepage} %Seitennummer
\renewcommand{\footrulewidth}{0.4pt}
\begin{document}

The deadline for this exercise sheet is \textbf{Monday, \thedate.}

\section*{Introductory Words}
In case we have some information that doesn't directly concern the current exercises.

\section{Task}
Lorem ipsum blah

\section{Another Task}
Lorem impsum blah part 2

\end{document}



\title{BPP Exercise 3 - Directing the Flow}
\author{A. Hain, M. Nipshagen}
\date{23.04.2018, 8:00}

\makeatletter
\let\thetitle\@title
\let\theauthor\@author
\let\thedate\@date
\makeatother

\pagestyle{fancy}
\fancyhf{}
\fancyhead[L]{\thetitle}
\fancyhead[C]{}
\fancyhead[R]{\theauthor}
\renewcommand{\headrulewidth}{0.4pt} %obere Trennlinie
\fancyfoot[L]{Due: \thedate}
\fancyfoot[R]{\thepage} %Seitennummer
\renewcommand{\footrulewidth}{0.4pt}

% include solutions
\renewcommand\sol[1]{#1}
% do not include solutions
% \newcommand\sol[1]{}

\begin{document}

The deadline for this exercise sheet is \textbf{Monday, \thedate.}
%
%\section*{Introductory Words}
%In case we have some information that doesn't directly concern the current exercises.
%
\section{Boolean Operators}
Determine the truth values of the following boolean operators for each configuration
of truth values as given in the tables

\subsection{}
\begin{tabular}{| c | c | c | c |}
  \hline
  \textbf{a} & \textbf{b} & \textbf{c} & \textbf{(b or c) and (a or c)} \\
  \hline
  true & true & true & \sol{true} \\
  \hline
  true & true & false & \sol{true} \\
  \hline
  true & false & true & \sol{true} \\
  \hline
  true & false & false & \sol{false} \\
  \hline
  false & true & true & \sol{true} \\
  \hline
  false & true & false & \sol{false} \\
  \hline
  false & false & true & \sol{true} \\
  \hline
  false & false & false & \sol{false} \\
  \hline
\end{tabular}

\subsection{}
\begin{tabular}{| c | c | c | c |}
  \hline
  \textbf{a} & \textbf{b} & \textbf{c} & \textbf{a or (b and c) and (not c or not a)} \\
  \hline
  true & true & true & \sol{false} \\
  \hline
  true & true & false & \sol{true} \\
  \hline
  true & false & true & \sol{false} \\
  \hline
  true & false & false & \sol{true} \\
  \hline
  false & true & true & \sol{true} \\
  \hline
  false & true & false & \sol{false} \\
  \hline
  false & false & true & \sol{false} \\
  \hline
  false & false & false & \sol{false} \\
  \hline
\end{tabular}



\section{Prof Strikes Again}

Write a function \texttt{passed} that takes a grade
as a parameter and returns whether the student passed or failed.

\cprotect\sol{
\begin{python}
def passed(grade):
  return grade <= 4.0
\end{python}
}


\subsection{}

\subsection{}

\end{document}
