\documentclass{article}

\usepackage{fancyhdr}


\title{BPP Exercise 1 - print("Hello World")}
\author{A. Hain, M. Nipshagen}
\date{09.04.2018, 8:00}

\makeatletter
\let\thetitle\@title
\let\theauthor\@author
\let\thedate\@date
\makeatother

\pagestyle{fancy}
\fancyhf{}
\fancyhead[L]{\thetitle}
\fancyhead[C]{}
\fancyhead[R]{\theauthor}
\renewcommand{\headrulewidth}{0.4pt} %obere Trennlinie
\fancyfoot[L]{Due: \thedate}
\fancyfoot[R]{\thepage} %Seitennummer
\renewcommand{\footrulewidth}{0.4pt}
\begin{document}

The deadline for this exercise sheet is \textbf{Monday, \thedate.}

\section*{Introductory Words}
In case we have some information that doesn't directly concern the current exercises.

\section{Task}
Lorem ipsum blah

\section{Another Task}
Lorem impsum blah part 2

\end{document}


\title{BPP Exercise 12 - Python and Experiments}
\author{A. Hain, M. Nipshagen}
\date{26.06.2018, 14:00}

\makeatletter
\let\thetitle\@title
\let\theauthor\@author
\let\thedate\@date
\makeatother

\newcommand\itemsub[1]{
	\begin{itemize}
		\item #1
	\end{itemize}
}

\setcounter{secnumdepth}{0}

\begin{document}
The deadline for this exercise sheet is \textbf{Tuesday, \thedate.}
\tableofcontents
\vspace{12pt}\noindent
\textbf{DISCLAIMER:} These are all just suggestions and not necessarily a complete
or the best approach to a solution. It just offers hints, general approaches
and ideas.\\
These are also a lot of pages of one-liners.
\pagebreak

\section{Dictionary and List}
The dictionary and list to use in the experiment could look like:
\begin{python}
exp_values = {1:(4.5, 128, 64), 2:(2.5, 128, 192), 3:(5, 192, -192)}
exp_idx = [1, 2, 3]
\end{python}
Those are not the complete dictionary and list, but should just show
you how the two are structured.

\pagebreak

\section{Getting a Random Key Value}
Once you have set up your dictionary and list, you can care about getting a random index.
How you do this depends on your framework, but here are two examples:\\
The \texttt{TrialHandler} object is able to hand over trials in a random fashion.\\
The \texttt{Block} object has the method \texttt{shuffle\_trials()}.

\pagebreak

\section{The Circle and the Cross}
Remember how we talked about clearing and flipping buffers? Clear the screen,
draw your stimuli, and only then flip / draw the whole scene.

\pagebreak

\section{Making the Pause}
Again there are several options here. Both frameworks have methods for time handling,
and so does Python. So you can either take the \texttt{Stopwatch} or \texttt{Clock}, or
use the \texttt{time} module.

\pagebreak

\section{Measuring Reaction Time}
And again. There are several options to deal with this. In \texttt{Expyriment} this does
not even need a single more line of code. Maybe have a look at the slides again.\\
In \texttt{PsychoPy} you might have to work around a bit more. You can either use 
\texttt{PsychoPy}'s time measuring capabilites or use Python's \texttt{time} module.
An idea would be to take a timestamp at stimulus onset (when the stimulus
is first shown) and when the key was hit by the user. Then take the difference of these
two to get the reaction time.


\end{document}
